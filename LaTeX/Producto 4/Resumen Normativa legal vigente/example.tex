\section{Normativa aplicable en materia de protección de datos.}
	Normativa aplicable en materia de protección de datos de carácter personal tanto en la legislación específica como en la sectorial.
	\subsection{Legislación actual.}
		\begin{itemize}
			\item REGLAMENTO (UE) 2016/679 DEL PARLAMENTO EUROPEO Y DEL CONSEJO de 27 de abril de 2016 relativo a la protección de las personas físicas en lo que respecta al tratamiento de datos personales y a la libre circulación de estos datos y por el que se deroga la Directiva 95/46/CE (Reglamento general de protección de datos) \href{https://www.boe.es/buscar/doc.php?id=DOUE-L-2016-80807}{Enlace}
			\item Corrección de errores del Reglamento (UE) 2016/679 del Parlamento Europeo y del Consejo, de 27 de abril de 2016, relativo a la protección de las personas físicas en lo que respecta al tratamiento de datos personales y a la libre circulación de estos datos y por el que se deroga la Directiva 95/46/CE (Reglamento general de protección de datos) \href{https://eur-lex.europa.eu/legal-content/ES/TXT/PDF/?uri=CELEX:32016R0679R(02)&from=ES}{Enlace}
			\item Real Decreto-ley 5/2018, de 27 de julio, de medidas urgentes para la adaptación del Derecho español a la normativa de la Unión Europea en materia de protección de datos. \href{https://www.boe.es/buscar/doc.php?id=BOE-A-2018-10751}{Enlace}
			\item Ley Orgánica 3/2018, de 5 de diciembre, de Protección de Datos Personales y garantía de los derechos digitales. \href{https://www.boe.es/buscar/doc.php?id=BOE-A-2018-16673}{Enlace}
		\end{itemize}
	\subsection{Legislación anterior.}
		\begin{itemize}
			\item Ley Orgánica 15/1999, de 13 de diciembre, de Protección de Datos de Carácter Personal (vigente en aquellos artículos que no contradigan el RGPD). \href{https://www.boe.es/buscar/act.php?id=BOE-A-1999-23750}{Enlace}
			\item Real Decreto 1720/2007, de 21 de diciembre, por el que se aprueba el Reglamento de desarrollo de la Ley Orgánica 15/1999 (vigente en aquellos artículos que no contradigan el RGPD). \href{https://www.boe.es/buscar/act.php?id=BOE-A-2008-979}{Enlace}
			\item Real Decreto 428/1993, de 26 de marzo, por el que se aprueba el Estatuto de la Agencia de Protección de Datos. \href{https://www.boe.es/buscar/doc.php?id=BOE-A-1993-11252}{Enlace}
		\end{itemize}
	\subsection{Sociedad de la información y telecomunicaciones.}
		\begin{itemize}
			\item Ley 34/2002, de 11 de julio, de servicios de la sociedad de la información y de comercio electrónico. \href{https://www.boe.es/buscar/act.php?id=BOE-A-2002-13758}{Enlace}
			\item Ley 9/2014, de 9 de mayo, General de Telecomunicaciones. \href{https://www.boe.es/buscar/act.php?id=BOE-A-2014-4950}{Enlace}
		\end{itemize}
\newpage
\section{Resumen de la Ley de Protección de datos actual.}
	La ley 15/1999, de Protección de Datos de Carácter Personal se aplica para para proteger los datos personales, libertades públicas y derechos fundamentales de las personas físicas. El ámbito de aplicación de esta ley incluye a todos los datos de carácter personal registrados en soporte físico y dichos datos no podrán usarse para finalidades diferentes de aquellas para las que se recogieron, debiendo ser cancelados cuando dejen de ser necesarios.
	
	\newp Para pedir los datos personales, lo primero que debemos hacer es informar a la persona que nos va a facilitar los datos de varias cosas:
	\begin{itemize}
		\item Los destinatarios de la recogida de dichos datos.
		\item La finalidad de la recogida de datos.
		\item Las consecuencias de la obtención.
		\item La identidad y dirección del responsable de su tratamiento.
	\end{itemize}
	
	\newp Siempre se va a requerir el consentimiento del afectado salvo que la recogida de datos sea  para el ejercicio de funciones propias de las Administraciones Públicas. No obstante, el consentimiento del afectado puede ser revocado en cualquier momento si se da una causa justificada.
	
	\newp Al respecto de la seguridad de los datos, la ley indica que el responsable del fichero deberá tomas las medidas de seguridad oportunas para garantizar su seguridad, además dichos responsables están obligados a cumplir una cláusula de secreto profesional con respecto a los mismos.
	
	\newp Así mismo, el registro general es público y puede ser consultado de forma gratuita por los usuarios para conocer los tratamientos de carácter personal, sus finalidades e identidad del responsable.
	
	\newp Si se cometen infracciones por respecto al tratamiento de datos personales, los afectados pueden interponer reclamaciones y tener derecho a indemnización cuando sufran daño o lesión sobre sus bienes.
	
	\newp Por otro lado es importante saber que toda persona que cede sus datos tiene derecho en cualquier momento a acceder, rectificar, opositar, o cancelar los datos referentes a su persona a través de los cauces oportunos.
\newpage
	\newp La Agencia de Protección de Datos se crea para la protección de los datos de carácter personal y asignándole funciones como:
	\begin{itemize}
		\item Velar por el cumplimiento de la legislación sobre protección de datos.
		\item Emitir las autorizaciones previstas en la Ley o en sus disposiciones reglamentarias.
		\item Indicar las instrucciones que deben seguir los tratamientos de datos.
		\item Atender las peticiones y reclamaciones de los afectados.
		\item Proporcionar información a las personas acerca de sus derechos.
		\item Ejercer la potestad sancionadora.
	\end{itemize}
	
	\newp La publicación del Real Decreto-ley 5/2018, de 27 de julio, de medidas urgentes para la adaptación del Derecho español a la normativa de la Unión Europea en materia de protección de datos ha supuesto un cambio respecto a las sanciones así como en los procedimientos a seguir  en caso de vulnerarse la normativa vigente en materia de protección de datos.
	
	\newp ¿Cuando es obligatoria la contratación de la figura de un Delegado de Protección de Datos (DPD)?. Los artículos 37 y 39 del Reglamento Europeo designan obligatoriedad de contratación en tres supuestos:
	\begin{enumerate}
		\item Si el tratamiento de los datos corre a cargo de una autoridad u organismo público.
		\item Si las actividades y operaciones principales del responsable de datos exigen seguimiento regular y sistemático a gran escala.
		\item Si las actividades y operaciones principales del responsable requieren tratamientos a gran escala de datos personales que tienen que ver con delitos y condenas.
	\end{enumerate}
	
\newpage
\section{Términos de nuestro proyecto que se ven afectados o podrían ser susceptibles de serlo.}
	Según indica la normativa actual, hay datos más sensibles que otros y son precisamente los más sensibles los que debemos proteger con mayor atención en pro de salvaguardar los datos personales de los usuarios, en este caso nuestro trabajadores. Nos referimos especialmente a datos como el número de la seguridad social, el DNI, el salario de nuestros empleados, dirección e incluso si está afiliado a un sindicato en particular, etc. Además, el cumplimiento de esta normativa ha de ser especialmente cautelosa en los casos de las empresas informáticas, debido a la enorme cantidad de datos sensibles que suelen gestionar, este tipo de organizaciones manejan una gran cantidad de datos de clientes. Cada vez que un cliente contrata el mantenimiento informático de sus sistemas o empresas exteriores ceden los datos de sus empleados para elaborar las nóminas, se está manejando datos de carácter personal de terceros. Para el correcto manejo de estos datos es preciso que se cumpla con la normativa de protección de datos.
	
	\newp Lo primero que debemos tener en cuenta es que para proceder a la recogida de datos es imprescindible que los propietarios de los datos personales autoricen explícitamente la recogida de los datos, así como que deben ser informados del uso que va a darse a los mismo y quien va a ser el responsable de ello.
	
	\newp Por otro lado, se intentará que los usuarios  que tengan acceso a los datos sensibles sean los mínimos posibles y que su vez deberán firmar un contrato con la entidad comprometiéndose a no hacer un uso indebido de los datos a los que tienen acceso.
	
	\newp Por supuesto, debemos también poner todas las medidas oportunas para impedir que los datos sean cedidos o vendidos a terceras personas y que estos sean usados con diferentes fines para los que fueron cedidos.
	
	\newp En nuestro proyecto, el usuario “admin” es quien tiene la capacidad en cualquier momento para acceder y poder ejercer los derechos del usuario de “acceder, rectificar, opositar, o cancelar los datos” referentes a cualquier otro usuario.
	
	\newp El usuario “empleado”, sin embargo, tendrá acceso a la actualización y visualización de la tabla empleado, así como a la visualización del resto de las tablas. No obstante, ya una vez desde la aplicación tendrá se le cortará el acceso para que solamente puedan ver los datos no sensibles de sus compañeros y que también solo se le permita la actualización de sus propios datos en la tabla empleado.
	
	\newp Y por último, el usuario “agencia” podrá eliminar, insertar, actualizar o visualizar los datos de todos los trabajadores de su agencia.
	
	
	\newp En cuanto a la conservación de los datos debemos tener en cuenta que siguiendo lo que nos dice la Ley de Protección de Datos con el concepto de la proporcionalidad, podremos guardar datos, pero solo si los necesitamos y durante el periodo de tiempo estrictamente necesario para realizar tus funciones, o lo que es lo mismo, mientras que los empleados lo sean.
	
	
	