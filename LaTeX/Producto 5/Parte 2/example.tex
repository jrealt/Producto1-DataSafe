\section{Realizar los siguientes disparadores.}
	\subsection{Un disparador que al añadir un nuevo empleado verifique que su apellido no puede tener una longitud superior a 15 caracteres.}
		\begin{sourcecodep}[]{c}{basicstyle={\fontsize{10}{10}\selectfont\ttfamily},firstnumber=1}{}
			DELIMITER //
			CREATE TRIGGER verificar_apellido BEFORE INSERT ON empleado FOR EACH ROW
			BEGIN
				declare longitud INT(10);
				declare mensaje_de_error VARCHAR(80);
				SET longitud = CHAR_LENGTH(NEW.apellido);
				IF (longitud > 15) THEN
					SET  mensaje_de_error = CONCAT('Verificar_Apellido: Error, apellido sobrepasa 15 carácteres. Longitud: ', CAST(longitud AS CHAR));
					SIGNAL sqlstate '45000' SET message_text = mensaje_de_error;
				END IF;
			END
			//\end{sourcecodep}
		\begin{sourcecodep}[]{c}{basicstyle={\fontsize{10}{10}\selectfont\ttfamily},firstnumber=1}{}
			INSERT INTO empleado (codigo_empleado, dni, nss, nombre, apellido, id_nombre_cat, id_central, id_ciudad_res, email, telefono_movil) 
				VALUES ('rpa11458', '89654725H', '4265987264', 'Raul', 'Perez Espinoza Mar', '3', '4', '1', 'raupe@gmail.com', '+34667526387'); \end{sourcecodep}
		\insertimageboxed[]{img/05}{scale=0.8}{0.5}{}
	\subsection{Un disparador que al añadir una cuota sindical, verifique que la cuota(número) es positivo.}
		\begin{sourcecodep}[]{c}{basicstyle={\fontsize{10}{10}\selectfont\ttfamily},firstnumber=1}{}
			DELIMITER //
			CREATE TRIGGER verificar_cuota_sindical BEFORE INSERT ON sindicato FOR EACH ROW
			BEGIN
				declare cuota FLOAT;
				declare mensaje_de_error VARCHAR(80);
				SET cuota = NEW.cuota;
				IF (cuota <= 0) THEN
					SET  mensaje_de_error = CONCAT('Verificar_Cuota_Sindical: Error, cuota sindical no es positiva. Cuota: ', CAST(cuota AS CHAR));
					SIGNAL sqlstate '45000' SET message_text = mensaje_de_error;
				END IF;
			END
			//\end{sourcecodep}
		\begin{sourcecodep}[]{c}{basicstyle={\fontsize{10}{10}\selectfont\ttfamily},firstnumber=1}{}
			INSERT INTO sindicato (nombre_sindicato, cuota) VALUES ('Ficticio', -40.5);\end{sourcecodep}
		\insertimageboxed[]{img/06}{scale=0.8}{0.5}{}
	\subsection{Un disparador que al eliminar una titulación, copie esta titulación en otra tabla (esta tabla deberá ser creada por vosotros).}
		\begin{sourcecodep}[]{c}{basicstyle={\fontsize{10}{10}\selectfont\ttfamily},firstnumber=1}{}
			CREATE TABLE IF NOT EXISTS titulaciones_eliminadas (
				codigo_empleado VARCHAR(8) NOT NULL,
				id_titulo INT NOT NULL,
				PRIMARY KEY (codigo_empleado),
				FOREIGN KEY (codigo_empleado) REFERENCES empleado (codigo_empleado) ON UPDATE CASCADE ON DELETE CASCADE,
				FOREIGN KEY (id_titulo) REFERENCES titulo (id) ON UPDATE CASCADE ON DELETE CASCADE);
			
			DELIMITER //
			CREATE TRIGGER salvaTitulacionesEliminadas BEFORE DELETE ON titulacion FOR EACH ROW
			BEGIN
				INSERT INTO titulaciones_eliminadas (codigo_empleado, id_titulo) VALUES (OLD.codigo_empleado, OLD.id_titulo);
			END;
			//\end{sourcecodep}
		\begin{sourcecodep}[]{c}{basicstyle={\fontsize{10}{10}\selectfont\ttfamily},firstnumber=1}{}
			DELETE FROM titulacion WHERE codigo_empleado LIKE 'crl38160';
			SELECT * FROM titulaciones_eliminadas;\end{sourcecodep}
		\insertimageboxed[]{img/07}{scale=1}{0.5}{}